% Author: Alfredo Sánchez Alberca (asalber@ceu.es)
\begin{tikzpicture}
  \begin{axis}[
  3dfun,
  height=5cm,
  xmin = -1, xmax=1,
  ymin=-1, ymax=1,
  zmin=0, zmax=2,
  axis equal = true,
  clip=false,
  color = myblack,
  ]
    \addplot3+[domain=0:pi, myblue, samples y=0] ({cos(deg(x))}, {sin(deg(x))}, {x});
    \addplot3+[domain=-pi/2:pi/2, myred, samples y=0] ({-x}, {1}, {x+pi/2});
    \coordinate (P) at (0,1,pi/2);
    \draw [->, mygreen, thick] (P) -- (-1,1,2.57) node[anchor=west, mygreen] {$\textbf{f}'(\pi/2)$};
    \fill (P) circle (1.2pt) node[anchor=west] {$P=\mathbf{f}(\pi/2)=(0,1,\pi/2)$};
    \node [myblue] at (0.3,0,-0.4) {$\mathbf{f}(t)=(\cos(t), \operatorname{sen}(t), t)$};
    \node [myred] at (1.4,1.4,1) {$(-t, 1, t+\pi/2)$};
  \end{axis}
\end{tikzpicture}
